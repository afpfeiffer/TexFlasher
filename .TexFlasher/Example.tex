\documentclass[10pt,a4paper]{article}
\usepackage[utf8x]{inputenc}
\usepackage[german]{babel}
\usepackage{amsmath, amsthm, amssymb}
\usepackage{amsfonts}
\usepackage{makeidx}
\usepackage{hyperref}
\usepackage{color} 
\usepackage[colorinlistoftodos]{todonotes} 

\newenvironment{fc}[1][optional]{}{}

\newtheorem{sat}{Satz}[section]
\newtheorem{bem}[sat]{Bemerkung}
\newtheorem{lem}[sat]{Lemma}
\newtheorem{defi}[sat]{Definition}
\newtheorem{ax}[sat]{Axiom}
\newtheorem{folg}[sat]{Folgerung}
\newtheorem{prop}[sat]{Proposition}


\newcommand{\TODO}[1]{\mbox{\fcolorbox{black}{red}{\textcolor{white}
{\textbf{TODO: }}}\fcolorbox{black}{white}{\textcolor{red}{\textbf{#1}}}}}
\let\Oldtodo\todo
\newcommand{\inlinetodo}[1]{ \TODO{#1} }
\renewcommand{\todo}[1]{\Oldtodo{#1} \\[0.2cm]}


\newcommand{\beachte}[1]{\parbox{300\unitlength}{\textbf{Beachte: }\textbf{#1}}}


\author{Can "Ozmen und Axel Pfeiffer}

\title{Zusammenfassungen}

\makeindex

\begin{document}

%###% end header 

%%%%%%%%%%%%%%%%%%%%%%%%%%%%%%%%%%%%%%%%%%%%%%%%%%%%%%%%%%%%%%%%%%%%%%%%%%%%%%%%%%%%%%%%%%%%%%%%%%%%%%%%%%%%%%%%%%%%%%%%%%%%%%%%%%%%


\section{This is a regular .tex file}
\subsection{Howto write Flashcards}

%fc=ExampleFlashCard
\begin{bem}[You have to fill in this field. It will be shown on the front of the flashcard.]
 Simply place the flashcard marker ``fc=\textbf{tag}'' in front of the theorem and 
 come up with a \textbf{tag} for it.
 The marker is invisible in the compiled pdf file.
 \index{Flashcard marker} % works fine!
\end{bem}






\printindex
\end{document}
